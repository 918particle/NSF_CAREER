\documentclass[../../main.tex]{subfiles}
 
\begin{document}
\label{sec:data_manage}

\textbf{Data Management Plan.} Given that a large portion of our proposed project involves open-source computational electromagnetism (CEM) software development, and the creation of a mobile application, our data management plan (DMP) is simple and straightforward.  Our general philosophy is to stay within open-source areas, ensuring that users can easily and efficiently access our codes, schematic files, and output data.  We must also ensure that the data generated by students using our mobile app is only educational and demographic, and does not contain private information.  We share below the six data products we anticipate creating through our project, and elaborate on the DMP components that will guide our actions if our proposed work goes forward.  We describe the data products we will generate, and the standard formats we will use.  We share how we will ensure appropriate access to our data products, and we describe our vision for the use of our data products.  Finally, we explain how we will archive the data in a way that preserves its utility for future research. \\ \vspace{2.5mm}

Our proposed project will produce six types of data products. (1) Python3 code and Jupyter notebooks, (2) schematics in GDSII format, (3) STL files for use in open-source CAD programs and 3D printing, (4) HDF5 output files, (5) RF testing and calibration data, and (6) Jupyter notebooks for educational and curricular purposes.  A hard product of our research will be scientific instrumentation created with additive manufacturing. \\ \vspace{2.5mm}

Our Python3 code will use the MIT Electromagnetic Propagation Package (MEEP) and machine learning packages like scikit-learn.  Depending on the use case, we will create this code in the form of scripts or Jupyter notebooks.  MEEP scripts have an enormous advantage for CEM, in that they can be run with parallel processing via MPI.  There are ways to run Jupyter notebooks in parallel.  It is often advantageous, however, to produce important results via MEEP scripts while providing well-documented examples in Jupyter notebooks.  Thus, the majority of our code will be in script form, accompanied as appropriate with demonstrations in Jupyter notebooks.  Jupyter notebooks can be extensively formatted for web-based usage with the Markdown language.  Thus, important metadata required to understand and use our code can be shared via the notebooks.  Both notebook files (.ipynb) and script files (.py) will be stored in an open, official GitHub repository, for transparency and accessibility. \\ \vspace{2.5mm}

The GDSII file format is a binary database which is the industry standard for Electronic Design Automation (EDA) data exchange of integrated circuit (IC) layouts.  The format represents planar geometric shapes, and other information organized in hierarchical form.  One use-case is to use GDSII (.gds) files to transfer layouts between IC design tools.  As part of our prior research, we have adapted the GDSII file format to RF antenna design after noting that GDSII designs can be imported into MEEP, which can also be used in RF design.  Given the standard nature of the GDSII file format, it is natural to include GDSII design files alongside our Python3 and Jupyter notebook files.  Users should be able to access our CAD designs so that our codes can be run in the same way we run them.  Individual GDSII files have a maximum size of $2^{16}$ bytes, or 65kB.  This means they should be stored alongside our codes in the open, transparent GitHub repository. \\ \vspace{2.5mm}

Using an open-source technique we discovered on GitHub, we can extrude two-dimensional GDSII layers into 3D ``stereolithography'' STL files.  STL files are one standard used for additive manufacturing.  Most 3D printing platforms can accept STL files as inputs.  Storing the STL files we generate, rather than assuming users can perform the GDSII to STL translation on their own, is a sensible choice.  The open nature of the STL format also ensures our project results can be printed by others in an open and transparent way.  Thus, we commit to storing our STL files alongside the GDSII, Jupyter notebook, and Python3 files in our open GitHub repository. \\ \vspace{2.5mm}

In some cases, it will be required to share the HDF5 file output from our MEEP and machine learning codes.  It will not be sufficient to provide only code and schematics for our most impactful CEM designs.  Users will want to check their results against ours.  The mission of the HDF Group is to ``ensure efficient and equitable access to science and engineering data across platforms and environments, now and forever.''  We agree with the importance of efficient and equitable access, but we have to specify \textit{how} we will ensure that access.  HDF5 output from MEEP can exceed size requirements for the git repository language, and would be inappropriate for a standard GitHub repository.  Luckily, there is precedent at Whittier College for sharing access to scientific data via the Whittier Domains platform.  For example, G. Piner and P.G. Edwards have organized a large catalogue of astrophysical data about TeV blazars, currently hosted as a Whittier Domains portfolio.  We think this is the appropriate venue to host our HDF5 file outputs from CEM calculations. \\ \vspace{2.5mm}

Our designs will be tested in the laboratory, and the testing data the fifth data product associated with the proposed work.  Given the limted size of this data product, it makes sense to store it in our GitHub repository alongside the design files and CEM code.  The only limiting factor for the git repository language is file size.  Our publication practice for our most recent publications has been to use git to track our LaTeX file templates for journal articles.  Thus, we envision a repository that contains our analysis, journal article writing, design files, and code in a common location, backed up locally at Whittier College.  In this way, users can determine how we draw the conclusions we share in our publications.  One peculiarity with RF testing data is that it often lacks metadata appropriate for its interpretation.  We commit to including (in associated Readme.md files) an explanation of the units, precision, and other metadata necessary to interpret the RF testing data.  \\ \vspace{2.5mm}

The educational Jupyter notebooks associated with the course integration portion of our proposal are most useful to the instructors of Whittier College courses.  These can be shared via Moodle, our content management system (CMS).  The Whittier College Moodle, however, requires access granted by the college.  Whittier Domains, however, does not.  Thus, Whittier Domains is the appropriate location to share learning modules designed for college courses that utilize the results of our research. \\ \vspace{2.5mm}

Using GitHub and Whittier Domains for storing and sharing our data products is that our group does not have to operate these sites.  Rather, we can manage our portfolios and repositories on these platforms periodically, respond to data sharing requests by sharing links to them, and disseminate knowledge of them through our publications.  We commit to citing these repositories in our future publications, as we have done in the past.  One example of this practice is ``Complex analysis of Askaryan radiation: A fully analytic treatment including the LPM effect and Cascade Form Factor,'' (Hanson, J.C. and Connolly, A.L., Astroparticle Physics, 2017).  In this publication, we created a C++ code that contained an analytic model we produced and studied.  Users were directed to the code repository from within the publication.  \\ \vspace{2.5mm}

The main focus of our project proposal is to facilitate an open-source ecosystem for RF design and production through additive manufacturing.  Thus, we will do what is necessary to ensure other groups can use our data products to the fullest extent.  Our formal distribution and re-use policy is informed by this.  Thus, we will ensure that all data products are accompanied by instructions and metadata necessary to reproduce our results.  From within our publications, we can provide the details of the additive manufacturing procedures that lead to success.  We plan to publish our open-source CEM and RF design results in the open access journal \textit{Electronics} (MDPI) in order to facilitate sharing these details.  This is especially important for Title-V HSIs like Whittier College that cannot devote the resources to subscribe to a wide variety of specialty engineering journals.  \\ \vspace{2.5mm}

Poet Commons for archival policy.  \\ \vspace{2.5mm}

\end{document}
