\documentclass[../../main.tex]{subfiles}
 
\begin{document}

\textbf{CAREER: Translation, Acceleration, and Diversification of Science and Engineering with Open-Source Computational Electromagnetism and Additive Manufacturing} \\ \vspace{2.5mm}

Radio-frequency (RF) phased-array systems optimized with machine learning have become powerful tools in science and engineering.  Phased-array radar development has applications in particle astrophysics \cite{Vieregg_2016,AVVA201746,electronics10040415,Aguilar_2021}, polar research \cite{arnold_2020,9670670}, and 5G mobile \cite{5G_review_paper}.  Phased-arrays are comprised of RF antennas working in tandem to boost received signal sensitivity, and to actively scan transmitted signals without moving parts.  Two pathways for progress in phased array design and production that will enhance future scientific work are cost reduction and the use of open-source software.  The electromagnetic properties of phased arrays are designed with expensive, proprietary software that does not interface with open-source machine learning tools \cite{10.3390/electronics8121506}.  They are then manufactured using costly and time-consuming traditional machining techniques.  Ongoing scientific and engineering efforts can be enhanced by a solution that allows machine learning to flourish, reduces design and manufacturing costs, and diversifies participation by reducing financial barriers.  This work will enhance STEM education with computational electromagnetism, machine learning, and 3D printing by integrating these topics into the curriculum.  \\ \vspace{2.5mm}

We propose to create the first open-source CEM and additive manufacturing ecosystem capable of 3D-printing phased arrays with conductive filament \cite{10.3390/electronics8121506, yurduseven,8786183}.  We have already shown that open-source CEM tools used in photonics can drive the RF phased-array design process \cite{electronics10040415,meepcon2022,10.1016/j.cpc.2009.11.008}.  This research will support diverse projects like IceCube Gen2 (radio), Center for Remote Sensing and Systems (CReSIS) missions, and Office of Naval Research (ONR) radar projects.  One application in particle astrophysics is the Askaryan Radio Array (ARA), in which phased arrays have increased sensitivity to ultra high-energy neutrino (UHE-$\nu$) interactions in the ice sheet beneath the South Pole \cite{PhysRevD.105.122006}.  The arrays are vertically polarized, due to mechanical constraints within the ice.  By combining machine learning with CEM, we seek a \textit{horizontally polarized} design that overcomes these mechanical constraints, boosting the chances of making the first UHE-$\nu$ observations in history \cite{10.1088/1748-0221/15/09/p09039}.  This research will \textit{accelerate} and \textit{diversify} research in UHE-$\nu$, climate science, and RF engineering by \textit{translating} successes in CEM and materials research.  This work will be integrated into our curriculum and research programming at Whittier College, a Title-V Hispanic Serving Institution (HSI).  This work applies to three of the 10 Big Ideas from NSF: Windows on the Universe, NSF INCLUDES, and Navigating the New Arctic. \\ \vspace{2.5mm}

This work will provide research and educational opportunities to diverse undergraduates at Whittier College.  We have a proud tradition of providing access to higher education to Spanish-speaking and historically marginalized students, and we are the only HSI member of the IceCube Gen2 collaboration.  People of color and first-generation students make up 63\% and 29\% of our student body, respectively.  Internal assessment studies indicate that students of color receive lower grades than their peers in introductory STEM courses.  We have learned from workshops hosted by the Cottrell Scholars Network that emphasizing the dignity and self-efficacy of diverse students can increase their performance \cite{cottrell1,cottrell2}.  Emphasis in these areas makes students feel they \textit{belong} in our courses, despite encountering adversity. In keeping with the theme of \textit{translation}, and in order to emphasize the dignity of our students no matter their background, we seek to create a bilingual (Spanish and English) mobile application (app) that introduces STEM concepts within a welcoming digital environment. \\ \vspace{2.5mm}

There is precedent for learning apps enhanced by machine learning in the Duolingo method for language and mathematics \cite{duolingo_whitepaper}. We seek to provide data insights about student learning to instructors through the app, which will lead to more efficient and customized classroom instruction.  A prototype application is being built by Whittier College undergraduates.  The creation and implementation of this program represents an opportunity for Whittier College students to enhance the learning experience for their peers while gaining valuable coding and machine learning experience.  In addition to algorithms presented within the Duolingo method, the educational data mining (EDM) literature provides examples of apps that boost engagement and success in introductory STEM courses \cite{edm1,edm2,edm3,edm4}.  Members of our community have shared that translating mathematics and physics exercises into Spanish aids in solving them.  Our application will boost their skills and build confidence by offering them engaging, game-like physics training in the language of their choice.  Finally, we propose to create a bilingual lecture series and recruitment events designed to welcome the broader community into the Whittier College environment.

\clearpage

\end{document}
