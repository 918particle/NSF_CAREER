\documentclass[../../main.tex]{subfiles}
 
\begin{document}

\section{Project Summary: Solicitation 22-586}

\textbf{Title: Translation of Machine Learning and Additive Manufacturing for a Diverse Scientific and Defense Workforce}

\textbf{Abstract (Intellectual Merits)}: Radio-frequency (RF) phased-array systems optimized with machine learning have become powerful tools in science and engineering.  Recent progress in phased-array radar development has applications in particle astrophysics, polar research, and defense.  Phased-array systems are comprised of RF antennas working in tandem to boost received signal sensitivity, and to actively scan transmitted signals without moving parts.  There are, however, at least two barriers that impede phased-arrays from enhancing future science and engineering projects.  First, the computational electromagnetism (CEM) properties of RF systems are designed with proprietary software that does not interface with common machine learning software.  Second, RF systems are manufactured using standard machine tools, which drives up costs.  We propose to create an open-source CEM and additive manufacturing ecosystem capable of 3D-printing phased array systems with conductive filament.  If successful, this research will reduce costs, boost sensitivity, and lower barriers to entry in a wide range of scientific contexts, including IceCube Gen2 (radio), CReSIS, and Office of Naval Research (ONR) defense projects.  This research has been supported by the ONR on a small scale, and recent results and publications indicate it is time to increase the scope of research and production.

\textbf{Abstract (Broader Impacts)}: Whittier College is a Title-V Hispanic Serving Institution (HSI), with a proud tradition of providing access to higher education to Spanish-speaking and traditionally under-represented students in Southern California and beyond.  Studies conducted by our Bayard Rustin Fellows indicate that our diverse students experience a variety of difficulties in introductory STEM courses.  Further, we have learned from inclusivity and STEM workshops hosted by the Cottrell Scholars Network that, in order to boost the success of diverse students, we must emphasize student dignity and self-efficacy.  These topics engage students in a way that makes them feel they will belong and thrive in our courses.  Within this context, we propose to create a free, bilingual (Spanish and English) mobile application that introduces STEM concepts within a digital environment that welcomes new students.  Further, our application will use machine learning techniques to adapt to individual students, and provide insights to optimize instructor performance.  Members of our community have shared experiences of translating mathematics problems into Spanish before solving them.  Our application would welcome such students by presenting new exercises in their first language.  We also propose two new bilingual lecture series.  First, we propose to create a bilingual physics lecture series, hosted at Whittier College and other community venues, that presents physics research to bilingual audiences.  Second, we propose a series of bilingual undergraduate engineering recruitment events designed to welcome new students and their families into our learning and research community.

\clearpage

\end{document}
