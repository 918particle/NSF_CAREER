\documentclass[../../main.tex]{subfiles}
 
\begin{document}

\textbf{Project Summary: Translation of Machine Learning and Additive Manufacturing to Accelerate and Diversify Science and Engineering} \\ \vspace{2.5mm}

Radio-frequency (RF) phased-array systems optimized with machine learning have become powerful tools in science and engineering.  Recent progress in phased-array radar development has applications in particle astrophysics \cite{Vieregg_2016,AVVA201746,electronics10040415,Aguilar_2021}, polar research \cite{arnold_2020,9670670}, and 5G mobile communications \cite{5G_review_paper}.  Phased-arrays are comprised of RF antennas working in tandem to boost received signal sensitivity, and to actively scan transmitted signals without moving parts.  There are at least two barriers that impede phased-arrays from enhancing future science and engineering projects on a wide scale.  First, the computational electromagnetism (CEM) properties of RF systems are designed with expensive, proprietary software that does not interface with open-source machine learning tools.  Second, RF systems are manufactured using traditional machining techniques that are time-consuming and expensive compared to additive manufacturing. \\ \vspace{2.5mm}

We propose to create an open-source CEM and additive manufacturing ecosystem capable of 3D-printing phased arrays with conductive filament, following recent efforts for single RF antennas designed with proprietary software \cite{8786183,yurduseven,10.3390/electronics8121506}.  We have already demonstrated that open-source CEM tools can drive the phased-array design process \cite{electronics10040415,meepcon2022,10.1016/j.cpc.2009.11.008}.  This research will reduce costs, boost performance, and lower barriers to entry in diverse applications ranging from IceCube Gen2 (radio), Center for Remote Sensing and Systems (CReSIS) missions, and Office of Naval Research (ONR) radar projects.  One example in particle astrophysics is the Askaryan Radio Array (ARA), in which phased arrays have increased sensitivity to ultra high-energy neutrinos (UHE-$\nu$) \cite{PhysRevD.105.122006}.  The societal impact of this research is to \textit{translate} successes in CEM and materials research in order to \textit{accelerate} progress across diverse disciplines of science and engineering. \\ \vspace{2.5mm}

Whittier College is a Title-V Hispanic Serving Institution (HSI), with a proud tradition of providing access to higher education to Spanish-speaking and traditionally under-represented students in Southern California and beyond.  Seventy percent of our undergraduates are people of color, and forty percent are first-generation students.  Studies conducted by our Bayard Rustin Fellows indicate that our diverse students experience a variety of difficulties in introductory STEM courses.  Disparities in introductory STEM courses is a well-known and complex problem \cite{doi:10.1146/annurev-soc-071312-145659}.  We have learned from workshops hosted by the Cottrell Scholars Network that emphasizing student dignity and self-efficacy can increase the performance of diverse undergraduates in our courses \cite{cottrell1,cottrell2}.  Emphasis in these areas makes students feel they \textit{belong} in our courses. \\ \vspace{2.5mm}

Within this context, we propose to create a bilingual (Spanish and English) mobile application that introduces STEM concepts within a digital environment that welcomes our students.  Our application will adapt to individuals using machine learning techniques, and provide data insights to instructors.  Our application is being built by Whittier College undergraduates, and is founded on precedent in the educational data-mining (EDM) literature \cite{edm1,edm2,edm3,edm4}.  Members of our community have shared that translating mathematics exercises into Spanish helps them solve them.  Our application would affirm the dignity and self-efficacy of such students by presenting new concepts in their first language.  Finally, we propose to create a bilingual physics lecture series and recruitment events designed to welcome broader community members into the Whittier College research and learning environment.

\clearpage

\end{document}
