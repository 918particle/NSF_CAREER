\documentclass[../../main.tex]{subfiles}
 
\begin{document}

\textbf{Project Summary: Translation of Machine Learning and Additive Manufacturing to Accelerate and Diversify Science and Engineering} \\ \vspace{2.5mm}

Radio-frequency (RF) phased-array systems optimized with machine learning have become powerful tools in science and engineering.  Recent progress in phased-array radar development has applications in particle astrophysics \cite{Vieregg_2016,AVVA201746,electronics10040415,Aguilar_2021}, polar research \cite{arnold_2020,9670670}, and 5G mobile communications \cite{5G_review_paper}.  Phased-arrays are comprised of RF antennas working in tandem to boost received signal sensitivity, and to actively scan transmitted signals without moving parts.  There are at least two barriers that impede phased-arrays from enhancing future science and engineering projects on a wide scale.  First, the computational electromagnetism (CEM) properties of RF systems are designed with expensive, proprietary software that does not interface with open-source machine learning tools \cite{10.3390/electronics8121506}.  Second, RF systems are manufactured using costly and time-consuming traditional machining techniques.  Ongoing scientific and engineering efforts can be enhanced by a solution that allows machine learning optimzation to flourish, reduces design and manufacturing costs, and diversifies participation by removing technical and financial barriers. \\ \vspace{2.5mm}

We propose to create an open-source CEM and additive manufacturing ecosystem capable of 3D-printing phased arrays with conductive filament, following recent efforts for single RF antennas designed with proprietary software \cite{10.3390/electronics8121506, yurduseven,8786183}.  We have already demonstrated that open-source CEM tools used in photonics can drive the RF phased-array design process by relying on the scale invariance of Maxwell's equations \cite{electronics10040415,meepcon2022,10.1016/j.cpc.2009.11.008}.  This research will support diverse, ongoing efforts ranging from IceCube Gen2 (radio), Center for Remote Sensing and Systems (CReSIS) missions, and Office of Naval Research (ONR) radar projects.  One application in particle astrophysics is the Askaryan Radio Array (ARA), in which phased arrays have increased sensitivity to ultra high-energy neutrino (UHE-$\nu$) interactions in the ice sheet beneath the South Pole \cite{PhysRevD.105.122006}.  ARA phased arrays are vertically polarized, due to mechanical constraints within the ice.  Our research could enhance ARA by discovering and adding a \textit{horizontally polarized} design that overcomes the mechanical constraints through machine learning.  This would enhance the precision of UHE-$\nu$ arrival direction and energy observations \cite{10.1088/1748-0221/15/09/p09039}.  Discovery of a UHE-$\nu$ flux would be a major breakthrough in particle astrophysics, and measuring accurately the neutrino interation properties will be key to that discovery.  The societal impact of this research is to \textit{accelerate} research in UHE-$\nu$, climate science, and RF engineering by \textit{translating} successes in CEM and materials research. \\ \vspace{2.5mm}

Whittier College is a Title-V Hispanic Serving Institution (HSI), with a proud tradition of providing access to higher education to Spanish-speaking and traditionally under-represented students in Southern California and beyond.  People of color and first-generation students make up 70\% and 40\% of our student body, respectively.  Studies conducted by our Bayard Rustin Fellows indicate that our diverse students experience a variety of difficulties in introductory STEM courses.  Disparities in introductory STEM courses is a well-known and complex problem \cite{doi:10.1146/annurev-soc-071312-145659}.  We have learned from workshops hosted by the Cottrell Scholars Network that emphasizing student dignity and self-efficacy can increase the performance of diverse undergraduates in our courses \cite{cottrell1,cottrell2}.  Emphasis in these areas makes students feel they \textit{belong} in our courses. In keeping with the theme of \textit{translation}, and in order to emphasize the dignity of our students no matter their background, we propose a service project designed to boost performance and engagement in introductory STEM courses at Whittier College and in our broader community. \\ \vspace{2.5mm}

We seek to create a bilingual (Spanish and English) mobile application (app) that introduces STEM concepts within a welcoming digital environment.  Our app will adapt to individuals using machine learning techniques.  There is precedent for such an app in the DuoLingo method for language and mathematics \cite{duolingo_whitepaper}. We seek to provide data insights about student learning to instructors through the app, which should lead to more efficient and customized classroom instruction.  A prototype application is being built by Whittier College undergraduates.  In addition to algorithms presented within the Duolingo method, the educational data mining (EDM) literature provides examples of apps that boost engagement and success in introductory STEM courses \cite{edm1,edm2,edm3,edm4}.  Members of our community have shared that translating mathematics and physics exercises into Spanish aids in solving them.  Our application would affirm their dignity and self-efficacy by offering them the ability to learn new concepts in their first language.  Finally, we propose to create a bilingual physics lecture series and recruitment events designed to welcome broader community members into the Whittier College research and learning environment.

\clearpage

\end{document}
