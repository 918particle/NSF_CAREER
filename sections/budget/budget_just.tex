\documentclass[../../main.tex]{subfiles}

I. Introduction
	i) Budget is crafted in coordination with professionals
		a) Plans account for current resources
		b) Plans provide for students, workers, and equipment
	ii) Budget reflects NSF guidelines for minimum and reasonable amounts
II. Personnel
	i) PI summer salary, no AY release
	ii) Academic year student researchers and translators
	iii) Summer researchers
	iv) Post-baccalaureate researchers
III. Equipment, Two Major Items
	i) GPU/Multi-core System76 server
	ii) Oscilloscope with $\geq 1.5$ GHz bandwidth
IV. Travel
	i) PI travel by train to regional conferences
	ii) Conference registration and online conferences
V. Other Direct Costs
	i) 3D printing
	ii) Misc. RF supplies
	iii) Phase shifters
	iv) Androids
	v) Drone parts
	vi) Publication fees
	vii) Invited speaker costs

 
\begin{document}
\label{sec:budget_just}

\textbf{Budget Justification.}  We have crafted a budget that contains the personnel, equipment, and travel expenses required by the proposed project.  Our budget has been generated with the help of our Associate Director of Research and Sponsored Programs, and it reflects fringe benefits and other indirect costs associated with Whittier College.  With the professional help and advice of our colleagues, we have created a budget that provides for the project needs while aligning with appropriate budget guidelines.  Namely, the final figure does not exceed the minimum dollar amount specified for CAREER proposals within the Directorate for Engineering (ENG) by more than 10\%.  We first provide justification for our budget items in the personnel category.  Second, we describe our major equipment items.  Third, we describe our travel expenses.  We conclude with our remaining direct costs, comprised of smaller equipment purchases.  \\ \vspace{2.5mm}

\textbf{Personnel.}  The majority of our budget request is in the form of personnel salary and wages.  Due to the small size of our Department of Physics and Astronomy, we are not requesting academic year course releases.  Rather, our project plans call for undergraduate researchers to work with us throughout Project Years 1-5 to accomplish our work.  Thus, the only PI salary in our budget is 2 person-months per year for Summary salary.  Our project planning calls for undergraduate researchers during the academic year (AY), and summer research internships.  The average number of AY students we hope to recruit is 4 per year, but this number is modified depending on the Project Year.  For example, our Project Plan calls for an additional undergraduate worker in Project Year 2 to act as Spanish language translator for our EASTLOS application (see Project Description for details).  In Project Year 4, certain projects have already concluded.  Thus, our budget only calls for two AY researchers in Semesters 7 and 8.  Regarding the undergraduate Summer fellowships, we typically assume three per year, but in Project Year 2 we only require one student to finish the initial work accomplished in Semesters 1 and 2.  By performing this analysis, we intend to minimize budgetary impact while completing our work. \\ \vspace{2.5mm}

In our personnel budget, we also include institutional wisdom gained after several years of mentoring undergraduates who have earned fellowships.  Whittier College is an undergraduate institution, meaning we lose the experienced juniors and seniors soon after we recruit them.  In other cases, first-year and sophomore recruits transfer to other institutions before a project finishes.  Thus, we have included ``post-baccalaureate'' researcher positions.  Paid at a higher rate, these recruits are meant to be recently graduated seniors looking to continue engineering work as their job search in the private sector progresses.  Other categories of students for these roles include students in our 3-2 engineering program, who leave Whittier College after three years, and students applying to graduate school. For students in these situations, we hope to retain talent for our project work, and to provide them with key experiences they will use in the workforce.  These students will be given some advisory and mentorship responsibilities as they work alongside their younger colleagues.  In this way, the budget reflects the organizational structure we will execute. \\ \vspace{2.5mm}

\textbf{Major Equipment Items.} Our budget calls for two major pieces of scientific equipment.  The first is a Tektronix mixed-domain oscilloscope (MDO) with bandwidth greater than 1.5 GHz.  This will allow us to visualize received signals from antennas-under-test (AUTs) in the time-domain.  Collecting data in the time-domain, for example, will allow us to discover reflected signals that represent faulty coaxial cable connections between the AUTs and transmitters.  Though the MDO can provide frequency domain data, we already have a network analyzer capable of providing spectral information over a 6 GHz bandwidth.  We will use the network analyzer to collect S-parameter and radiation pattern data in the frequency domain.  Thus, the exploratory quotes we have obtained for the MDO focus on time-domain precision, number of RF channels, and measurement features in the time-domain.  The main driver of cost for MDOs is bandwidth, which must be $\geq 1.5$ GHz for our proposed designs. \\ \vspace{2.5mm}

The next major equipment item reflects the need for computational resources.  There are two possible upgrades to our facilities that can achieve this goal.  First, we could purchase a new System76 GPU/multi-core server in either of the Jackal or Ibex product lines.  This new system would be used for machine learning and CEM calculations in service to our Project Plan, and for new computational research by our colleagues and students.  We envision a central server that will allow faculty from the Depts. of Physics and Astronomy, and Mathematics and Computer Science to perform computational work with job-scheduling.  Thus, we could utilize our System76 Thelio (see Facilities, Equipment, and Other Resources documentation) for initial CEM runs, and the shared server for calculations that require more power.  Prof. Fred Park (current chair of Computer Science and Mathematics Dept.) has the experience to utilize a GPU server for projects that would complement our research.  This approach bolsters our efforts to integrate our research and educational endeavors, for students could obtain server accounts used for course work in Machine Learning or Introduction to Data Science with Python. \\ \vspace{2.5mm}

Alternatively, we could use the computational resources in the budget request to upgrade our System76 Thelio.  CEM calculations that use the FDTD approach are memory-intensive.  We could advance our CEM work with a memory upgrade to the Thelio, because this would enable more student accounts to run simultaneous jobs.  This approach would keep the computational work on our private desktop, meaning only students in our group would benefit.  However, this approach is simple and straightforward, requiring no server maintenance.  With either approach, the open-source CEM portion of our Project Plan will be accomplished.  Course integrations using heavy-duty computations could still take place, but in a more limited fashion for the courses taught by Prof. Hanson. \\ \vspace{2.5mm}

\textbf{Travel Expenses.}  To present our work to the physics, engineering, and geoscience communities, we are targeting specific conferences in online and in-person formats.  Our Project Plan calls for online attendance of the International Cosmic Ray Conference (ICRC).  In 2025, ICRC will be held in Geneva, Switzerland.  We have already presented online at ICRC 2021, with no technical issues.  Our budget requests for conference fees reflect our intent to attend ICRC 2025 (and potentially ICRC 2027) in the online format.  We also target domestic conferences such as April Meeting of the American Physical Society (APS), or the Annual Meeting of the American Geophysical Union (AGU) in San Fransisco (2026).  We plan to use Amtrak and other ground transportation, when feasible.  In the case of the AGU meeting in San Fransisco (2026), Amtrak transportation between Los Angeles and San Fransisco is fiscally and logistically efficient.  We will also give at least one presentation at a RF/microwave engineering conference.  Our budget reflects this general plan for conferences. \\ \vspace{2.5mm}

\textbf{Other Direct Costs.} In the category of Other Direct Costs, we include smaller hardware items, and other fees and expenses.  First, we spread the cost of ten 3D printers over Project Years 1-5.  We investigated the price of the Prusa MK4 line of printers as our baseline.  We seek to purchase ten printers for two reasons.  The first is that we would like to integrate this fleet of printers into our curriculum, allowing students to gain the engineering experience in future courses.  The second reason is that we are attempting to print \textit{phased arrays}, meaning we will print multiple RF antenna elements simultaneously.  This approach allows us to print an array of the appropriate size, in parallel.  In addition, the printers can serve as backups to one another in case a printer needs maintenance.  We choose to spread the cost of the printers over Project Years 1-5 in order to gain experience with the chosen model before proceeding with more purchases.  Our second budget request in this category is the Electrifi conductive 3D printer filament.  We elected to spread the cost of the Electrifi rolls over Project Years 1-5, to prevent rolls from expiring if exposed to moisture when not being used. \\ \vspace{2.5mm}

As with any RF lab, we need a small yearly budget for RF coaxial cables, connectors, and filters.  We have used Pasternack, Inc., a local RF company, for such purchases in the past.  Another part available at Pasternack is the adjustable RF phase shifter.  Compatible with coaxial cable setups, these components will allow us to do beam forming and beam steering with our RF antennas acting as a phased array.  This will allow us to confirm array behavior without purchasing expensive RF integrated circuits.  This approach also avoids spending time and money on firmware design.  Firmware associated with the phased arrays we develop is inevitably going to be re-developed by other groups, and we felt this approach is cleaner. \\ \vspace{2.5mm}

For the development and testing of our EASTLOS app for Android Mobile, we plan to acquire a batch of mobile devices running Android OS.  This will allow us to test our application in a real-world setting.  As we are aware of the various regulations surrounding the purchase of mobile phones, we do not intend to purchase data plans for these devices.  Rather, we will rely on Whittier College WiFi and USB connections to test our app functionality. \\ \vspace{2.5mm}

In Project Years 3-5, we include budget requests for drone parts.  These include parts like carbon fiber tubing as structural components, quad-rotor motors, transceivers, GPS antennas, and small solar panels.  The portion of our 3D printing budget already includes material for hull construction, and we have 3D printed drone hulls in the past.  We already have the machine shop tools and soldering station in our labs (see Facilities, Equipment, and Other Resources).  Thus, we are minimizing budgetary impact, while planning to construct at least one new drone designed with RF phased array via open-source CEM. \\ \vspace{2.5mm}

Finally, we include resources for the publication of results from our intellectual and service activities.  We are targeting at least two journals: \textit{Electronics} (MDPI), and \textit{The American Journal of Physics.} In our analysis, we use the usual journal fees for \textit{Electronics}, converted from CHF to USD.  We have chosen this journal because it is open-access.  This boosts access for our students and our institution, which cannot purchase a wide variety of engineering journals.  We also discovered this line of research from a review article in \textit{Electronics,} so it is a natural choice.  For the analysis of our educational data collected with the EASTLOS app, we are targeting \textit{The American Journal of Physics}, published by the American Association of Physics Teachers (AAPT).  The focus of this journal is physics education.  Part of our proposed service activity is the organization of our bilingual STEM lecture series.  We include a small speaker honorarium and travel expenses for these events.  We seek to use these resources to attract quality speakers from institutions within and outside the United States, in order to enrich the educational experience of our students. \\ \vspace{2.5mm}

\end{document}
