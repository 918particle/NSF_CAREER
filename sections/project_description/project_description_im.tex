\documentclass[../../main.tex]{subfiles}
 
\begin{document}

Radio-frequency (RF) phased arrays have applications in radar telemetry, telecommunications, ground-penetrating radar, scientific instrumentation, and remote sensing \cite{Vieregg_2016,AVVA201746,arnold_2020,PhysRevD.105.122006,10.3390/s21186091,10.1016/j.jappgeo.2022.104876,phased_array_book}.  In the one-dimensional case, $N$ three-dimensional RF antennas are arranged in a line with fixed spacing.  In the two-dimensional case, $N \times M$ three-dimensional antenna elements are arranged in a two-dimensional grid with fixed spacing in both dimensions.  The signal to noise ratio (SNR) of received signals in arrays of dimension $N$ is boosted by a factor of $\approx \sqrt{N}$, because the $N$ signals are combined coherently while thermal noise adds like $\sqrt{N}$.  The SNR boost is critical for certain kinds of scientific observations.  For example, systems created at the Center for Remote Sensing and Integrated Systems (CReSIS) are flown in polar regions to perform radar sounding of ice sheets for the purposes of geophysics and climate science \cite{arnold_2020}.  Reflected signals carry information about the ice depth, temperature, and internal structure of the ice.  The radio echoes have small SNR values that require phased arrays.  \\ \vspace{2.5mm}

Traditionally, RF phased arrays are designed with commerical computational electromagnetism (CEM) software.  Radio antennas and phased arrays have \textit{radiation patterns} that define directions of maximum transmission power and received sensitivity.  Radiation patterns have a main lobe in which most of the radiation is concentrated, and the angular width of the main lobe is called the beam width.  Other parameters like S-parameters quantify the efficiency of the systems.  CEM packages like XFDTD and HFSS are used to model these properties as a function of frequency \cite{remcom,ansys}.  The XFDTD package, for example, relies on the finite difference time domain (FDTD) method. The FDTD approach is a CEM technique in which spacetime and Maxwell’s equations are broken into discrete form.  HFSS uses a similar approach in the Fourier domain, the Finite Element Method (FEM).  Depending on the software license and version, the current price of these products ranges between \$5,000 and \$40,000 USD.  These costs are prohibitive for HSI undergraduate institutions like Whittier College.  Removing this financial barrier to entry would enable diverse researchers to gain important skills in the field of RF design. \\ \vspace{2.5mm}

Another drawback of commercial CEM software is the lack of access to the source code, which impedes the incorporation of modern machine learning packages.  Phased array properties are determined by the shape of the RF elements and the grid properties of the array, and the parameter space is driven by the complex variety of RF element shapes.  When combined with open-source CEM software, modern machine learning algorithms can locate optimal solutions within the parameter space.  The authors of \cite{10.3390/electronics8121506} review a number of open-source CEM packages, and conclude that there are viable open-source options for simple RF antenna shapes.  For our proposed work, the open-source CEM software must be able to handle the growing complexity of RF antenna designs.  One interesting choice is the MIT Electromagnetic Equation Propagation (MEEP) package \cite{10.1016/j.cpc.2009.11.008}.  Though MEEP was designed for $\mu$m wavelengths in photonics applications, we have shown that the scale-invariance of Maxwell's equations allows MEEP users to translate designs to wavelengths at the cm-scale.  We have also shown that MEEP can drive the RF phased-array design loop, and that 3D printer schematics can be extracted from this process \cite{electronics10040415,meepcon2022,10.1016/j.cpc.2009.11.008}.  This research represents an opportunity for diverse undergraduates to gain experience applying machine learning tools to the design of practical systems.  \\ \vspace{2.5mm}

Recent advances in materials research have led to the creation of 3D printer filament that has conductivity in RF bands.  Funded through an NSF Translational Impact (TI) award (1721644), Multi3D LLC. has produced filament with a resistivity of just $10^{-2} \Omega$ cm: the Electrifi filament.  Several antenna designs have already been produced \cite{8786183,10.1049/iet-map.2017.0104}.  These examples include horn antennas with gain factors of 15 dB at 5.8 GHz, and microstrip patch antennas with gains of 1-2 dB at 2.5 GHz.  The results match expectations from HFSS models, exhibiting no major differences with antennas made using perfect conductors.  There are, however, virtually no examples of 3D printed RF phased arrays in the [0.1 - 1] GHz bandwidth.  This bandwidth is the most relevant for the aforementioned applications in particle astrophysics and geophysics.  Further, whole new designs can be discovered that improve on designs like the horn and patch antennas by merging machine learning packages with MEEP.  In Sec. \ref{sec:cem}, we review progress already made at Whittier College.  In Sec. \ref{sec:askaryan}, we show how this work enhances the field of UHE-$\nu$ observations.  In Sec. \ref{sec:cresis}, we show how this work enhances the field of radio echo sounding of ice sheets and ice shelves.  In Sec. \ref{sec:int}, we articulate our vision for the integration of this research into our STEM curriculum at Whittier College.  In Sec. \ref{sec:time_im}, we provide a general project timeline, broken into manageable phases.  In Sec. \ref{sec:time_im}, we make the case that the merits of our proposed project phases are sound, and useful to multiple fields.

\section{Computational Electromagnetism and Additive Manufacturing}
\label{sec:cem}

In Summer 2020, we received a Faculty Fellowship from the Office of Naval Research (ONR) to study and design phased arrays in the [0.1 - 5] GHz bandwidth.  This bandwidth is relevant for projects like IceCube Gen2 (radio), and Whittier College is a member institution of the IceCube Gen2 collaboration.  With our background in NSF-funded projects like the Antarctic Ross Ice Shelf Antenna Neutrino Array (ARIANNA), the Askaryan Radio Array (ARA), and NASA-funded projects like the Antarctic Impulsive Transient Antenna (ANITA), we were qualified to teach our ONR colleagues about phased array applications.  Our goal was to design a phased array system to be integrated as a transmitter in an anechoic chamber.  The anechoic chamber will serve as a testing facility for active radar systems.  We began by giving lectures on the electromagnetism of phased arrays and scientific and engineering applications.  The audience included engineers and programmers that work in acquisition and development for the Naval Surface Warfare Center (NSWC), Corona Division (NSWC Corona).  Our design flow is depicted in Fig. \ref{fig:design}.  To minimize costs and increase access to Whittier College students, we decided to investigate open-source CEM options for the design. \\ \vspace{2.5mm}

\begin{figure}
\centering
\includegraphics[width=0.85\textwidth]{diagram3.pdf}
\caption{\label{fig:design}  Our design process for RF phased arrays from \cite{electronics10040415}, adapted from Fig. 1 of the review \cite{10.3390/electronics8121506}.}
\end{figure}

We encountered the aforementioned review article in the open-access journal \textit{Electronics} that indicated there are open-source CEM tools that can be adapted to phased array analysis.  Our design flow in Fig. \ref{fig:design} is adapted from Fig. 1 of the review to include specific tasks required for phased arrays, and algorithms for the computation of far-field radiation patterns.  MEEP was noted by the authors in the review as the most advanced among open-source FDTD programs, but they did not benchmark it against HFSS or XFDTD due to the ``steep'' learning curve.  As part of the ONR Summer Faculty Fellowship, we ascended the learning curve and adapted MEEP to RF systems.  The key insight was that MEEP takes advantage of the \textit{scale invariance} of Maxwell's Equations.  The simplest way to understand this is to understand how MEEP uses relative units when discretizing Maxwell's equations for Python code. \\ \vspace{2.5mm}

Like other FDTD CEM methods, MEEP uses a Yee lattice to discretize Maxwell's equations \cite{10.1109/tap.1966.1138693}.  When the speed of light is set to unity ($c = 1$), distance and time units are set to be the same.  Frequency and wavelength units are the inverse of each other.  But distance and wavelength can take \textit{any} unit of length in the Yee lattice.  Most MEEP users interpret this unit of length to be 1 $\mu$m because the applications are for photonics.  For example, a \textit{relative} frequency (unit-less) of 0.5 corresponds to a \textit{relative} wavelength of 2.  When interpreted as 2 $\mu$m, the frequency is 150 THz in real units that correspond to optical bandwidth.  If we choose to interpret the \textit{relative} wavelength as 2 cm, the real frequency is 15 GHz.  A \textit{relative} frequency of 0.05 corresponds to the RF frequency 1.5 GHz.  Assuming design components have sufficient conductivity at RF frequencies, we have re-purposed MEEP as an RF simulator.  \\ \vspace{2.5mm}

\begin{figure}
\centering
%\includegraphics[width=0.35\textwidth]{figures/Oct30_plot2.png}
\includegraphics[width=0.33\textwidth]{figures/Oct30_plot1.png}
\includegraphics[width=0.33\textwidth]{figures/Aug11_plot2.png}
\includegraphics[width=0.33\textwidth]{figures/Aug11_plot1.png}
\caption{\label{fig:pa_1} (Left) The beam angle $\Delta \phi$ divided by the beam width $BW$ for the $N = 16$ one-dimensional Yagi array versus $\Delta \Phi$, the phase shift per element. The gray line represents theoretical expectation, and the black line is a linear fit to the data.  (Middle) $\Delta \phi$ versus $\Delta \Phi$ for the $N=16$ version of the one-dimensional horn array, for several frequencies.  (Right) The dependence of the beam width on frequency for the one-dimensional $N=16$ horn array.  The black line is a functional fit to the data $f(x) = a/x + b$ with $a=12.0\pm 0.1$ degree GHz, and $b=1.1\pm 0.2$ degrees.}
\end{figure}

By Fall 2020, we were producing CEM models using MEEP that matched expected phased array properties.  For a one-dimensional array with $N$ elements, there is a linear relationship between the radiated plane-wave direction $\Delta \phi$, and the phase shift per element $\Delta \Phi$.  The coefficient of the relationship is determined by the ratio of real wavelength to element spacing.  Figure \ref{fig:pa_1} contains results for our first phased array models in which the elements were Yagi-Uda style antennas and horn antennas.  The linear relationship is evident in the data.  The radiated signal direction $\Delta \phi$ is divided by the beam width (BW) in Fig. \ref{fig:pa_1} (left), and is left in degrees in Fig. \ref{fig:pa_1} (middle).  A beam width of a radiation pattern is the angular width of the main lobe, outside of which the radiated power has decreased by 3 dB.  In Fig. \ref{fig:pa_1} (left), the $N=16$ Yagi array can steer a 5 GHz plane wave up to four beam widths to the right or left of the forward direction.  Yagi-Uda style antennas are designed for a single frequency.  In Fig. \ref{fig:pa_1} (middle), results are shown for an $N=16$ array of horn antennas.  Since horn antennas are broadband radiators, the linear relationship is shown for 0.3, 1.5, and 3.0 GHz.  The beam width is inversely related to frequency, so $\Delta \phi$ was left in degrees.  In Fig. \ref{fig:pa_1} (right), the inverse relationship is shown. \\ \vspace{2.5mm}

We can also produce phased array radiation patterns with MEEP that match theoretical expectations.  The radiation pattern of a one-dimensional array of $N$ radiating point sources can be derived using first principles \cite{electronics10040415}.  The \textit{pattern multiplication theorem} states that the radiation pattern of a one-dimensional phased array of $N$ identical elements will be that of a row of $N$ point sources, multiplied by the radiation pattern of the individual element.  In Fig. \ref{fig:1dhornresults2} (left and middle), the radiated field of a $N=16$ horn array is shown in the E-plane (x-y plane).  The radiation pattern is shown in \ref{fig:1dhornresults2} (right).  The main lobe is steered 9 degrees above the x-axis, matching the theoretical expectation.  The blue curve in the polar plot represents the CEM radiation pattern from MEEP, while the red curve is the theoretical expectation from a row of $N$ point sources.  The row of point sources is symmetric, creating a back lobe at $\Delta \phi = 171$ degrees.  The horn array has no back lobe because the individual horns suppress backward radiation, as expected from the pattern multiplication theorem.  We also showed that two-dimensional arrays of Yagi-Uda and horn antennas matched theoretical expectations exactly.  Our revelation that the photonics code MEEP could used to design phased arrays design earned the final article Top 10 honors for December 2020 to May 2021 from the editors of \textit{Electronics}. \\ \vspace{2.5mm}

\begin{figure}
\centering
\includegraphics[width=5.625cm,angle=90]{figures/fields/colorbar.pdf}
%\includegraphics[width=3cm]{figures/fields/ey_phase_horn_t15.png}
\includegraphics[width=3cm]{figures/fields/ey_phase_horn_t30.png}
%\includegraphics[width=3cm]{figures/fields/ey_phase_horn_t45.png}
\includegraphics[width=3cm]{figures/fields/ey_phase_horn_t60.png}
\includegraphics[width=6cm]{figures/fields/rad_patt_field.png}
\caption{\label{fig:1dhornresults2} (Left) The $N = 16$ one-dimensional horn array, radiating a linearly polarized electric field $\vec{E}(x,y,t)$ (y-component shown, in arbitrary units) at $t = 1$ ns into the simulation run, and (middle) at $t = 2$ ns into the run.  The 2D area is $80 \times 150$ cm$^2$.  The frequency is 2.5 GHz, and the beam angle is $\Delta \phi = 9$ degrees from broadside (x-direction). (Right) The normalized radiated power in dB versus $\Delta \phi$.  The blue curve represents the results from MEEP, and the red curve is the theoretical expectation from $N$ point sources.}
\end{figure}

\begin{figure}
\centering
\includegraphics[width=0.35\textwidth]{figures/blender_example.png}
\includegraphics[width=0.3\textwidth]{figures/3dprinter.jpg}
\includegraphics[width=0.3\textwidth]{figures/3dprinter_2.jpg}
\caption{\label{fig:3d_print} (Left) Blender/STL files extracted from MEEP code.  (Middle) MakerBot 3D printer, with PLA horn model (white), and  proto-pasta with SMA connector (black). (Right) Close-up of horns.}
\end{figure}

\begin{figure}[hb]
\centering
\includegraphics[width=0.65\textwidth]{figures/multi3dllc.png}
\caption{\label{fig:3d_print2} Resistivity results published by Multi3D LLC that compare the proto-pasta product with the new Electrifi conductive filament (\url{https://www.multi3dllc.com/faqs/}).}
\end{figure}

In Summer 2021, we again received a Faculty Fellowship from the ONR to continue this work.  We focused on creating realistic 3D models of horn antennas that could be printed with 3D printers.  Working with undergraduate researchers, we learned to create designs that can be expressed as Python3 functions and converted to a GDSII CAD file.  GDSII files can be imported into MEEP, and converted to STL files for use with a 3D printer.  Our CEM codes are therefore using the precise shape that we intend to print.  We acquired NinjaTek proto-pasta 3D printer filament, advertised as conductive.  We printed a horn with in-built SMA connecter for RF cables (Fig. \ref{fig:3d_print}). The proto-pasta result had the right shape, but a measured resistance too large for an RF antenna.  Multi3D LLC, the manufacturer of the Electrifi filament, has now provided resistivity results that compare proto-pasta with Electrifi (Fig. \ref{fig:3d_print2}).  The Electrifi filament will improve resistivity by two orders of magnitude.  We seek to print Electrifi-based antennas, and to measure the radiation pattern and S-parameters. \\ \vspace{2.5mm}

In Summer 2022, we received a final ONR Faculty Fellowship that focused on GPS M-code and modernization.  Alongside this work, we continued to refine the open-source RF horns.  This included computing radiation patterns and S-parameters for the full 3D horns stored in CAD files.  In Fig. \ref{fig:3d_cad} (a), the main lobes are designed to point to 0 degrees (x-direction) for the E-plane (x-y plane), and 90 degrees for the H-plane (x-z plane).  The E-plane is the plane containing the linearly polarized radiation vector, and the H-plane is orthogonal to the E-plane.  In Fig. \ref{fig:3d_cad}, the (voltage standing wave ratio) VSWR is shown.  The VSWR is a common figure of merit for RF antennas, related to the S11 scattering parameter.  The VSWR approaches 1 for an efficiently radiating antenna.  The radiation patterns match expectations for horn antennas (see Fig. 19 of \cite{8786183}).  The VSWR results demonstrate efficient radiation in the bandwidth [0.5 - 6] GHz.  We presented our progress at the annual MeepCon 2022 at the Massachusetts Institute of Technology (MIT) \cite{meepcon2022}.  We learned the extent to which MEEP can be integrated with Python3-based machine learning tools \cite{meepcon2022_2}, and how eager MEEP developers are to collaborate in the RF regime.  Extending MEEP to RF users widens the user base of MEEP, which has traditionally focused on photonics applications. \\ \vspace{2.5mm}

\begin{figure}
\centering
\begin{subfigure}{0.65\textwidth}
    \includegraphics[width=0.49\textwidth]{figures/3DHorn_CAD_0_5GHz_E_plane.png}
	\includegraphics[width=0.49\textwidth]{figures/3DHorn_CAD_5GHz_E_plane.png} \\
	\includegraphics[width=0.49\textwidth]{figures/3DHorn_CAD_0_5GHz_H_plane.png}
	\includegraphics[width=0.49\textwidth]{figures/3DHorn_CAD_5GHz_H_plane.png}
    \caption{Radiation pattern results using GDSII/CAD for (top left) E-plane at 0.5 GHz, (top right) E-plane at 5.0 GHz, (bottom left) H-plane at 0.5 GHz, (bottom right) H-plane at 5.0 GHz.  See text for details.}
\end{subfigure}
\hfill
\begin{subfigure}{0.3\textwidth}
    \includegraphics[width=0.99\textwidth]{figures/vswr.png}
	\caption{The VSWR figure of merit versus frequency in GHz for the RF horn.}
\end{subfigure}
\caption{Results for RF horn design, using the open-source design process open to 3D printing.}
\label{fig:3d_cad}
\end{figure}

\subsection{RF Laboratory Capability and Prior ONR Funding}

As of May 2023, we have officially begun an Educational Partnership Agreement (EPA) between NSWC Corona and Whittier College.  As part of the EPA, NSWC Corona has the ability to transfer laboratory equipment to Whittier College.  NSWC Corona has provided RF bench testing equipment that is perfectly suited to the proposed work (see Tab. \ref{tab:equip}).  Our network analyzer and power sensors can perform S-parameter measurements over a bandwidth that encompasses the proposed bandwidth of [0.1 - 5 GHz] for our antennas under test (AUT).  Our signal generator can create calibration signals for our calibration antennas and AUT in the proposed bandwidth.  Our systems come with calibration antennas that serve as benchmark devices for comparison to the AUT we create via 3D printing.  Due to the precision and wide bandwith of these devices, they require calibration.  Our calibration kits serve this purpose.  Our laboratory is therefore already outfitted for the RF testing and calibration required to complete the proposed projects.  This minimizes the impact of new equipment in our proposed budget. \\ \vspace{2.5mm}

According to ONR policy, Faculty Fellows who have received fellowships three years in a row must take a mandatory gap year.  In Summer 2024, we will be eligible for Senior Faculty Fellowships because we received regular Faculty Fellowships in the Summers 2020-22.  Our EPA contacts with NSWC Corona have indicated that our focus will broaden to include cybersecurity and reliability engineering projects.  These projects will be in the form of student internships, curricular enhancements, and summer research projects.  Given the connections to particle astrophysics and geophysics, it is wise to organize the proposed work under the NSF CAREER program, so that it will maintain this focus.  We provide a summary of funding received related to this project in Tab. \ref{tab:funds}.  Note the regular involvement of undergraduate researchers.  These researchers have diverse majors and interests, including our 3-2 Engineering Program (Wildanger), Physics and Math double major (Hartig), and Math/Integrated Computer Science (G\'{o}mez-Reed) and Householder), and Physics and Astronomy (Goodman and Smith). \\ \vspace{2.5mm}

\begin{table}
\centering
\begin{tabular}{c c c}
Equipment & Bandwidth & Cost \\ \hline
Rohde and Schwartz ZVL6 Network Analyzer & 9 kHz to 6 GHz & \$20k \\
Rohde and Schwartz NRP-91 Power Sensors (2) & 9 kHz to 6 GHz & \$8k \\
Aeroflex 3416 Digital RF Signal Generator & 250kHz to 6 GHz & \$12k \\
Calibration antenna kits (2) & Varies by antenna & \$2k \\
Calibration test kits for Network Analyzer (2) & 6 kHz to 9 GHz & \$6k
\end{tabular}
\caption{\label{tab:equip} A listing of the equipment provided to our labs by the Office of Naval Research.}
\end{table}

\begin{table}
\centering
\begin{tabular}{c c c c}
Student/Professor & Grant Opportunity & Amount & Dates \\ \hline
Jordan C. Hanson & ONR Summer Faculty Fellow & \$16.5k & Summer 2022 \\
Dane Goodman & Summer researcher & Course credit & Summer 2022 \\
Andrew Householder & Summer researcher & Course credit & Summer 2022 \\
Raymond Hartig & Ondrasik-Groce Fellowship & \$5k & Summer 2022 \\
Jordan C. Hanson & ONR Summer Faculty Fellow & \$16.5k & Summer 2021 \\
Adam Wildanger & Fletcher Jones Fellowship & \$5k & Summer 2021 \\
Jordan C. Hanson & ONR Summer Faculty Fellow & \$16.5k & Summer 2020 \\
Raymond Hartig & Fletcher Jones Fellowship & \$5k & Summer 2020 \\
John Paul G\'{o}mez-Reed & Ondrasik-Groce Fellowship & \$7.5k & Summer-Fall 2019 \\
John Paul G\'{o}mez-Reed & Keck Fellowship & \$5k & Summer 2018 \\
Cassady Smith & Keck Fellowship & \$5k & Summer 2018 \\
\end{tabular}
\caption{\label{tab:funds} A listing of the grant opportunities awarded to our group for RF design, softrware development, and machine-learning.  All students are at the undergraduate level.}
\end{table}


\section{The Connection to Ultra-high Energy Neutrino Observations}
\label{sec:askaryan}

The flux of neutrinos with energies between [0.01-1] PeV has been detected by IceCube \cite{10.1126/science.1242856}.  The UHE-$\nu$ flux could potentially explain the origin of UHE cosmic rays (UHECR), and represents an opportunity to study electroweak interactions at record-breaking energies \cite{Ackermann:201946d,Ackermann:20195ec}.  Previous analyses have shown that the discovery of UHE-$\nu$ above 1 PeV will require an expansion in detector volume, because the UHE-$\nu$ flux is expected to decrease with energy \cite{10.1016/j.astropartphys.2010.06.003,10.1088/1475-7516/2010/10/013,10.1103/physrevd.98.062003,10.1088/1475-7516/2020/03/053,10.1103/physrevd.102.043021}.  Whereas the current version of IceCube detector observes neutrinos via optical signals that travel $<100$ m, the Askaryan effect translates a UHE-$\nu$ interaction into an RF pulse that travels more than 1 km in RF-transparent media such as Antarctic and Greenlandic ice \cite{10.3189/2015jog14j214, 10.3189/2015jog15j057, 10.1016/j.astropartphys.2011.11.010}.  Utilizing the Askaryan effect therefore allows for detectors with vastly larger effective volumes than optical observations. \\ \vspace{2.5mm}

The Askaryan effect occurs when a neutrino with velocity larger than the speed of light in a dielectric medium initiates a high-energy cascade with negative total charge.  The charge radiates energy in the RF bandwidth \cite{askaryan1,zhs}.  The IceCube EHE analysis has constrained the UHE-$\nu$ flux to be $E_\nu^2 \phi_\nu \leq 2 \times 10^{-8}$ GeV cm$^{-2}$ s$^{-1}$ sr$^{-1}$ between $[5\times 10^{15} - 2\times 10^{19}]$ eV \cite{10.1103/physrevd.98.062003}.  Arrays of $\mathcal{O}(100)$ \textit{in situ} detectors encompassing effective areas of $\approx 10^4$ m$^2$ sr per station, spaced by $\mathcal{O}(1)$ RF attenuation length could discover a UHE-$\nu$ flux beyond the EHE limits.  Polar ice formations in Antarctica and Greenland have the longest attenuation lengths.  A group of prototype Askaryan-class detectors has been deployed in polar regions that seek to probe unexplored UHE-$\nu$ flux parameter-space \cite{rice,10.1088/1475-7516/2020/03/053,10.1103/physrevd.102.043021,10.1103/physrevd.99.122001}. \\ \vspace{2.5mm}

Askaryan radiation was first observed in laboratory settings \cite{saltzberg,10.1103/PhysRevD.74.043002,ask_ice}.  The RF pulse shape is influenced by the shape of the cascade, and the observed pulse is strongest when viewed close to the Cherenkov angle.  Working with an undergraduate researcher, we recently published a theoretical model of the electromagnetic field of Askaryan radiation \cite{PhysRevD.105.123019}. Askaryan models are incorporated into simulations like AraSim in order to calculate expected signals and aid in detector design \cite{dookayka2011characterizing,testbed,10.1140/epjc/s10052-020-7612-8}.  For example, software developed for IceCube Gen2 (radio) utilizes machine learning and the Askaryan pulse shape to provide a way to reconstruct UHE-$\nu$ properties in future data \cite{10.1140/epjc/s10052-019-6971-5,10.1088/1748-0221/15/09/p09039,IFT}.  Askaryan electromagnetic fields are combined with RF channel responses to form ``signal templates'' used to search large data sets for signal candidates \cite{10.1088/1475-7516/2020/03/053,10.1016/j.astropartphys.2014.09.002}.  Data sets are large for Askaryan-class detectors due to the inevitable RF thermal background data.  For Askaryan signals, the SNRs at RF channels are expected to be small (SNR $\approx 3$), because the amplitude of the Askaryan field decreases with the distance to the UHE-$\nu$ interaction, and the signal is attenuated by the ice \cite{10.3189/2015jog14j214,Barwick:2018497,ALLISON201963}.  Template-waveform matching between models and data is a powerful technique for isolating high-energy particles \cite{10.1016/j.astropartphys.2015.04.002,10.1016/j.astropartphys.2014.09.002,barwick2016radio}. \\ \vspace{2.5mm}

Phased arrays have been incorporated into Askaryan-class prototype detectors, after it was noted that phased arrays would enhance the probability of observing UHE-$\nu$ \cite{Vieregg_2016,AVVA201746}.  Phased arrays boost the SNR of observed signals by combining multiple observations of the same RF pulse coherently.  Examples of this strategy are ARA5 \cite{PhysRevD.105.122006}, and the first deployments of Radio Neutrino Observatory, Greenland (RNO-G) \cite{rno}.  The arrays in each are identical, vertically polarized dipole antennas.  These decisions were made for mechanical reasons, because the array must fit in a 100 m deep, vertically-drilled borehole in the ice.  The radiation pattern exhibits azimuthal symmetry, and there is no sensitivity to the horizontal Askaryan field component.  Further, the designs assume a uniform index of refraction for the ice surrounding the array.  As part of our proposed work, we seek to use machine learning to discover horizontally polarized array designs that fit into the borehole and account for the index of refraction, $n$.  \\ \vspace{2.5mm}

We included a short study of phased array behavior in the South Pole ice environment in our recent publications \cite{electronics10040415,meepcon2022,10.1016/j.cpc.2009.11.008}.  Most commercial CEM packages assume a uniform $n$ in the surrounding medium.  By contrast, MEEP gives the user 3D control of the index of refraction, $n(x,y,z)$.  The RF index of refraction varies with the depth ($z$) near the snow surface.  The $n(z)$ function is well-measured in a variety of locations in Antarctica \cite{horizPaper}, and Greenland \cite{deaconu_2018}.  ARA (South Pole) \cite{PhysRevD.105.122006}, RNO-G (Greenland) \cite{rno}, and IceCube Gen2 (radio) (South Pole) \cite{Aartsen_2021} can all benefit from designs that account for $n(z)$ and have sensitivity to the horizontal component of the Askaryan field. \\ \vspace{2.5mm}

The common simulation package used for ARA, RNO-G, and IceCube Gen2 is NuRadioMC, built from prior experience with ARA and ARIANNA \cite{10.1140/epjc/s10052-020-7612-8,10.1109/tns.2015.2468182,10.1016/j.astropartphys.2011.11.010,Barwick:2014pca,10.1103/physrevd.102.043021}.  NuRadioMC addresses analytically the ray-tracing solution for UHE-$\nu$ signals as they propagate through polar ice.  We derived the analytic ray-tracing solutions presented in \cite{10.1140/epjc/s10052-020-7612-8} and \cite{horizPaper}, which were adopted into NuRadioMC.  A goal of our proposed research will be to incorporate realistic, 3D field propagation into NuRadioMC using FDTD computations with MEEP, with our analytic Askaryan model as the MEEP source \cite{PhysRevD.105.123019,10.22323/1.395.1217}.  This integration should boost the accuracy of the computations made with NuRadioMC, which will be matched with future ARA, RNO-G, and IceCube Gen2 data to isolate UHE-$\nu$ signals.

\section{The Connection to Remote Sensing of Ice Sheets}
\label{sec:cresis}

A gap exists in Askaryan-based UHE-$\nu$ science.  A knowledge of the RF attenuation length, $\lambda$, versus frequency, depth, and location is paramount to understanding UHE-$\nu$ detector sensitivity.  Although we have made detailed measurements of $\lambda$ versus frequency \cite{10.3189/2015jog14j214,10.3189/2015jog15j057,barwick_besson_gorham_saltzberg_2005}, we do not scan detector volumes to measure this parameter versus geographic location: $\lambda(x,y)$.  Further, $\lambda(z)$ is merely inferred from the average attenuation and ice core temperature data (see Fig. 24 from \cite{10.1016/j.astropartphys.2011.11.010}). IceCube Gen2 (radio) will require $\lambda(x,y,z)$ to be measured precisely.  CReSIS radio sounding data, available on the Open Polar Server (OPS: \url{https://ops.cresis.ku.edu/}), have been used to constrain $\lambda(x,y)$ across Greenland \cite{10.1002/2015rs005849}.  Far less CReSIS data is available near the location of IceCube (South Pole), due to the complex logistics of organizing flights in that region.  A new technological effort to incorporate radio sounding instrumentation into unmanned aerial systems (UAS) is underway, and it can benefit geophysics and particle astrophysics. \\ \vspace{2.5mm}

UAS systems offer a way to enrich radio sounding data for geophysics and partricle astrophysics.  OPS radio sounding data is generated from human-piloted fixed-wing aircraft with straight flight lines that carry on-board radar.  Flight lines can be hundreds of kilometers long, scanning wide areas with synthetic aperature radar (SAR) techniques.  There are, however, three key disadvantages.  First, there may not be a flight near the desired location, which is the case for South Pole.  Second, flights only give a snapshot of the ice at the time, and aircraft may not return to the same location for several years.  Third, the bandwidth of the radar does not always overlap with the bandwidth desired for IceCube Gen2 (radio).  Dedicated UAS could constrain $\lambda(x,y)$ in both the temporal and spatial regimes.  UAS are able to hover and fly at lower altitudes, so they can collect a wider variety of data than traditional fixed-wing craft.  For example, the CReSIS ultra-wide band (UWB) Snow Mini radar system was integrated onto the AeroVironment Vapor 55 UAS.  The low altitude flight capability increases the SNR in difficult areas by pushing clutter angles outside the field of view.  The SNR is also boosted by hovering due to increased integration time over a single site  \cite{arnold_2020}. The average cost of the Vapor 55, however, is about \$$90$k USD.  \\ \vspace{2.5mm}

In our RF design lab at Whittier College, our group has already constructed a 3D printed drone using PLA, carbon-fiber tubing, commercial motors, and commerical transmitters.  The unit has $\approx 1$ kg payload, a 20-min flight time, and is powered by LiPo batteries, for a total cost of $\approx 1$k USD (see Fig. \ref{fig:drone}).  Before the onset of the COVID-19 pandemic paused in-person laboratory work, we had plans to equip it with solar charging and cold-temperature components.  Thus, there is a potential for collaboration between CReSIS and IceCube Gen2 (radio) to solve a common problem: the solar rechargeable $\lambda(x,y)$ measurement system with vertical take-off and landing (VTOL).  Our drone design can be 3D printed and assembled from commercial parts for \$$\approx1$k with VTOL, but we need valuable insights from the CReSIS group on retro-fitting for cold temperatures and solar charging.  When outfitted with a 3D-printed phased array radar, we will have a formidable system capable of filling gaps in our knowledge of polar ice sheets.  \\ \vspace{2.5mm}

\begin{figure}
\centering
\includegraphics[width=0.6\textwidth]{drone.jpg}
\caption{\label{fig:drone} Our 3D-printed quad-rotor drone, designed and assembled by Whittier College undergraduates using our RF design lab and machine shop.  The unit is equipped with hand-held RC control, and GPS with programmable waypoints.}
\end{figure}

Phased array radio sounding systems mounted on UAS will be highly useful for polar research, but these designs face engineering limitations that must be overcome.  The optimization of craft weight, thrust, payload, and flight time is the primary problem to be solved.  To collect quality radio sounding data, the payload must be flown horizontally for $1-10$ km, implying $\approx 1$ hr battery life at reasonable speeds.  The requirement for longer flight times drives up battery size.  Increased battery size increases weight, which tends to decrease flight time.  Phased array payloads with a large number of elements could benefit data collection, but increased payload adds weight and decreases flight time.  The optimization is made far easier if the radar transmitter and receiver system is integrated into the hull of the UAS.  We propose to study how the RF phased array can be printed into the hull of the UAS, using machine learning to optimize beam-forming for radio sounding.  The Electrifi filament has a similar density to aluminum, meaning it can serve as \textit{both} a structural component, and phased array.  Manufacturing structural components as phased array elements reduces costs and weight.  This idea would reduce payload, thereby facilitating the effort to minimize radar units for UAS integration.  \\ \vspace{2.5mm}

As part of this engineering effort, we also propose to simulate expected results using MEEP.  Performing a CEM simulation that incorporates our current knowledge of ice properties with the radar response would enrich the research in two ways.  First, such simulations enhance the design process, revealing design requirements, shortcomings, and ways to overcome them.  MEEP simulations require the user to specify the complex matrix for the dielectric constant of the medium, $\epsilon(x,y,z)$.  The $\epsilon$ matrix  determines how RF waves reflect, refract, and propagate back to the receiver.  Optimizing UAS phased array design for maximal $\epsilon(x,y,z)$ precision will result in optimal precision for $\lambda(x,y,z)$.  For dielectric materials, $\epsilon$ and $\lambda$ are related analytically \cite{10.3189/2015jog14j214}.  Second, matching MEEP simulation output to data collected in the field will provide a cross-check between the observed $\epsilon(x,y,z)$ and the simulated $\epsilon(x,y,z)$. \\ \vspace{2.5mm}

FDTD simulations are notorious for consuming computational resources like volatile memory, while providing the necessary resolution for $\epsilon(x,y,z)$, and electromagnetic fields versus time, frequency, and space.  At Whittier College, we have acquired a System76 Helio with AMD Ryzen threadripper 3990x 64-core, 128 thread processor.  The system has 0.5 GB of volatile memory per thread.  We have already shown that running MEEP in parallel on our system reduces run times by an order of magnitude \cite{meepcon2022}.  The reduction is due primarily to increased set up speed for $\epsilon(x,y,z)$.  Thus, we are already in a position to perform such CEM simulations quickly and efficiently.  Learning how to introduce parallelism into CEM problems will also be of educational benefit to our STEM undergraduates at Whittier College.

\section{Integration of Research and Education at Whittier College}
\label{sec:int}

If our proposed work goes forward, the integration of CEM and additive manufacturing into our STEM curriculum will benefit our diverse undergraduates in two ways.  First, we plan to integrate examples from our research into existing courses.  Second, we will create undergraduate research opportunities.  Some obvious CEM course integrations can be done wihtin our Department of Physics and Astronomy.  Other integrations can take place in the Department of Mathematics and Computer Science.  We already have mathematics and computer science faculty at Whittier College who specialize in machine learning.  We will work with those colleagues to integrate this research as an application of machine learning into the appropriate courses.  Our undergraduate researchers have made wonderful achievements with our local Ondrasik-Groce and Fletcher Jones Fellowships.  Our goal is to expand this practice through NSF-sponsored research fellowhips in additive manufacturing and machine learning. \\ \hspace{2.5mm}

%1. Course integrations for computational physics, electromagnetism, data science (COSC180 - Glenn)
%What courses exist that can be enhanced?  Jupyter as a learning mechanic
%2. Course integrations for machine learning and additive manufacturing
%Do we need a new course here?  Or just the math ones?  How can we merge additive manufacturing into a course
%3. What type of undergraduate research fellowship is appropriate?  Summers are the most productive.
%Highlight past funding and show that we can at least continue on that trajectory, and if it is enhanced, the
%project can succeed.

One straightforward example is to incorporate CEM tools like MEEP into lower and upper division electromagnetism and Python3 courses.  Our 3-2 Engineering Program students, Physics majors, and Integrated Computer Science (ICS) majors all stand to benefit from learning to use Python to perform computational physics.  Our current curriculum does not yet include CEM in lower or upper-division electromagnetism courses, nor is it included in computational physics.  Integrating results from this research into course management systems, via MEEP Jupyter notebooks, is a straightforward way to enhance STEM education for our diverse undergraduates.  Showcasing the 3D printed RF systems should engage their curiosity by providing a real-world application of course concepts.  Finally, given the diverse demographics of our students, enriching their educational experience with real-world applications serves to diversify the STEM workforce.  We propose to develop project-based learning (PBL) modules that incorporate RF design, machine-learning, and additive manufacturing for our Whittier College STEM students.  \\ \vspace{2.5mm}

\section{Timeline and Project Planning, Intellectual Merits}
\label{sec:time_im}

Example

\end{document}
