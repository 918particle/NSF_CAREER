\documentclass[../../main.tex]{subfiles}
 
\begin{document}

Radio-frequency (RF) phased array systems are examples of active electronically-scanned arrays (AESAs) that have applications in radar telemetry, telecommunications, ground-penetrating radar, scientific instrumentation, and remote sensing \cite{edm1}.  All AESAs have \textit{radiation patterns} that define directions of maximum transmission power and received sensitivity, both due to constructive interferance of the transmitted or received RF plane wave.  In the one-dimensional case, $N$ three-dimensional RF antennas are arranged in a line with fixed spacing.  In the two-dimensional case, $N \times M$ three-dimensional antenna elements are arranged in a two-dimensional grid with fixed spacing in both dimensions.  Further complexity can be introduced in the grid structure to refine the radiation pattern.  The signal to noise ratio (SNR) of received signals in one-dimensional phased arrays of dimension $N$ is boosted by a factor of $\approx \sqrt{N}$, relative to a single RF antenna.  With fixed grid spacing, there is a linear relationship between the plane-wave phase difference in adjacent antennas and the incoming plane wave direction.

%Radio-frequency phased array antenna systems with design frequencies of order 0.1–10 GHz have applications in 5G mobile telecommunications, ground penetrating radar (GPR) systems, and scientific instrumentation [1–4]. In the one-dimensional case, a series of three-dimensional antenna elements are arranged in a line with fixed spacing [5]. Common antenna designs like loops and dipoles can be used to limit the elements to two dimensions. In this special case, phased array radiation may be modeled in two spatial dimensions plus time. In the two-dimensional case, a series of three-dimensional antenna elements are arranged in a two-dimensional pattern, often a grid with fixed element spacing in both dimensions. The elements may be strictly two-dimensional, but there is still an increase in computational complexity and the radiation is calculated in three dimensions plus time. Proprietary RF modeling packages like XFDTD and HFSS are often used to model the response of elements within phased arrays and the behavior of arrays [6–9]. The XFDTD package, for example, relies on the finite difference time domain (FDTD) method. The FDTD approach is a computational electromagnetics (CEM) technique in which spacetime and Maxwell’s equations are broken into discrete form. One variant of the FDTD method is the conformal FDTD method (CFDTD), recently used to study phased array concepts on a large scale [9]. The NEC2 and NEC4 family of codes relies on the method-of-moments (MoM) approach [10]. Aside from the cost, a drawback of proprietary modeling software can be a lack of fine control over each individual object in the simulation. Because Maxwell’s equations are scale-invariant, in principle open-source FDTD codes designed for optical regimes could be re-purposd for RF design workflows. One such open-source package is the MIT Electromagnetic Equation Propagation (MEEP) package [11]

\section{Computational Electromagnetism and Additive Manufacturing}

Example

\section{The Connection to Ultra-high Energy Neutrino Observations}

Example

\section{The Connection to Remote Sensing of Ice Sheets}

Example

\section{The Connection to Office of Naval Research Projects}

Example

\end{document}
