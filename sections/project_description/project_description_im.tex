\documentclass[../../main.tex]{subfiles}
 
\begin{document}

Radio-frequency (RF) phased arrays have applications in radar telemetry, telecommunications, ground-penetrating radar, scientific instrumentation, and remote sensing \cite{Vieregg_2016,AVVA201746,arnold_2020,PhysRevD.105.122006,10.3390/s21186091,10.1016/j.jappgeo.2022.104876,phased_array_book}.  In the one-dimensional case, $N$ three-dimensional RF antennas are arranged in a line with fixed spacing.  In the two-dimensional case, $N \times M$ three-dimensional antenna elements are arranged in a two-dimensional grid with fixed spacing in both dimensions.  The signal to noise ratio (SNR) of received signals in arrays of dimension $N$ is boosted by a factor of $\approx \sqrt{N}$, because the $N$ signals are combined coherently while thermal noise adds like $\sqrt{N}$.  The SNR boost is critical for certain kinds of scientific observations.  For example, systems created at the Center for Remote Sensing and Integrated Systems (CReSIS) are flown in polar regions to perform radar sounding of ice sheets for the purposes of geophysics and climate science \cite{arnold_2020}.  Radio signals transmitted from aircraft propagate downward through the ice.  Reflected signals carry information about the ice depth, temperature, and the presence of internal layers.  The radio echoes have small SNR values that require phased array receivers.  \\ \vspace{2.5mm}

Traditionally, phased array systems used in scientific projects are designed by hand, and commerical software is purchased to create the designs.  Radio antennas and phased arrays have \textit{radiation patterns} that define directions of maximum transmission power and received sensitivity.  Radiation patterns usually have a main lobe in which most of the radiation is concentrated, and the angular width of the main lobe is called the beam width.  Commercial computational electromagnetism (CEM) packages like XFDTD and HFSS are used to model properties of phased arrays like the radiation pattern \cite{remcom,ansys}.  The XFDTD package, for example, relies on the finite difference time domain (FDTD) method. The FDTD approach is a CEM technique in which spacetime and Maxwell’s equations are broken into discrete form.  HFSS uses a similar approach in the Fourier domain, the Finite Element Method (FEM).  Depending on the software license and version, the current price of these products ranges between \$5,000 and \$40,000 USD.  These costs are prohibitive for Title-V HSI undergraduate institutions like Whittier College.  Removing this financial barrier to entry would enable diverse researchers to participate in the design process. \\ \vspace{2.5mm}

Aside from the cost, a drawback of commercial modeling software is the lack of access to the source code.  This impedes the incorporation of machine learning packages within RF design.  Phased array system properties can be optimized to a given application using machine learning.  These properties are determined by the shape of the RF elements and the grid properties of the array.  The parameter space is large and complex, but machine learning tools can locate optimal solutions if they can interface with the CEM software.  The authors of \cite{10.3390/electronics8121506} review a number of open-source CEM packages, and conclude that there are viable open-source options for simple RF antenna shapes.  The open-source CEM software must be able to handle the growing complexity of RF antenna designs.  One interesting choice is the MIT Electromagnetic Equation Propagation (MEEP) package \cite{10.1016/j.cpc.2009.11.008}.  Though MEEP was designed for $\mu$m wavelengths in photonics applications, we have shown that the scale-invariance of Maxwell's equations allows MEEP users to translate designs to wavelengths at the cm-scale.  We have also shown that MEEP can drive the RF phased-array design process, and that 3D printer schematics can be extracted from this process \cite{electronics10040415,meepcon2022,10.1016/j.cpc.2009.11.008}. \\ \vspace{2.5mm}

Recent advances in materials research have led to the creation of 3D printer filament that has conductivities relevant for RF antenna production.  Resulting from an NSF Translational Impact (TI) award (1721644), Multi3D LLC. has produced filament with a resistivity of just 0.006 $\Omega$ cm: the Electrifi filament.  Several antenna designs have already been produced \cite{8786183,10.1049/iet-map.2017.0104}.  These examples include horn antennas with gain factors of 15 dB at 5.8 GHz, and microstrip patch antennas with gains of 1-2 dB at 2.5 GHz.  The results match expectations from HFSS models, exhibiting no major differences with antennas made using perfect conductors.  There are, however, virtually no examples of 3D printed RF phased arrays in the [0.1 - 1] GHz bandwidth.  This bandwidth is the most relevant for the aforementioned applications in particle astrophysics and geophysics.  Further, the 3D printed results that take advantage of the Electrifi filament from Multi3D LLC appear to be well-known designs.  These designs can be optimized, and whole new designs can be discovered, by merging machine learning packages like Scikit-Learn with MEEP.  In Sec. \ref{sec:cem}, we review progress already made at Whittier College.  In Sec. \ref{sec:askaryan}, we show how this work enhances the field of UHE-$\nu$ observations.  In Sec. \ref{sec:cresis}, we show how this work enhances the field of radio echo sounding of ice sheets and ice shelves.  In Sec. \ref{sec:onr}, we show how this work can benefit active radar testing.  In Sec. \ref{sec:conc_im}, we make the case that the overall intellectual merits of the proposed activities are sound.

\section{Computational Electromagnetism and Additive Manufacturing}
\label{sec:cem}

In Summer 2020, we received a Faculty Fellowship from the Office of Naval Research (ONR) to study and design phased arrays in the [0.1 - 5] GHz bandwidth.  This bandwidth is relevant for projects like IceCube Gen2 (radio), and Whittier College is a member institution of the IceCube Gen2 collaboration.  With our background in NSF-funded projects like the Antarctic Ross Ice Shelf Antenna Neutrino Array (ARIANNA), the Askaryan Radio Array (ARA), and NASA-funded projects like the Antarctic Impulsive Transient Antenna (ANITA), we were qualified to teach our ONR colleagues about phased array applications.  Our goal was to design a phased array system to be integrated as a transmitter in an anechoic chamber.  The anechoic chamber will serve as a testing facility for active radar systems.  We began by giving lectures on the electromagnetism of phased arrays and scientific and engineering applications.  The audience included engineers and programmers that work in acquisition and development for the Naval Surface Warfare Center (NSWC), Corona Division (NSWC Corona).  Our design flow is depicted in Fig. \ref{fig:design} below.  As the COVID-19 pandemic took hold, in-person work and funding for new engineering development both stopped.  We then made the decision to investigate open-source options for the CEM phase of the design. \\ \vspace{2.5mm}

\begin{figure}
\centering
\includegraphics[width=0.85\textwidth]{diagram3.pdf}
\caption{\label{fig:design}  Our design process for RF phased arrays from \cite{electronics10040415}, adapted from Fig. 1 of the review \cite{10.3390/electronics8121506}.}
\end{figure}

We encountered the aforementioned review article in the open-access journal \textit{Electronics} that indicated there are open-source CEM tools that can be adapted to phased array analysis.  Our design flow in Fig. \ref{fig:design} is adapted from Fig. 1 of the review to include specific tasks required for phased arrays, and algorithms for the computation of far-field radiation patterns.  MEEP was noted by the authors in the review as the most advanced among open-source FDTD programs.  The authors of the review did not benchmark it against HFSS or XFDTD due to the ``steep'' learning curve.  As part of the ONR Summer Faculty Fellowship, we ascended the learning curve and adapted MEEP to RF systems.  The key insight was that MEEP takes advantage of the \textit{scale invariance} of Maxwell's Equations.  The simplest way to understand this is to understand how MEEP uses relative units when breaking Maxwell's equations into usable statements in algorithms and code. \\ \vspace{2.5mm}

Like other FDTD CEM methods, MEEP uses a Yee lattice to discretize Maxwell's equations \cite{10.1109/tap.1966.1138693}.  When the speed of light is set to unity ($c = 1$), distance and time units are set to be the same.  Frequency and wavelength units are the inverse of each other.  But distance and wavelength can take \textit{any} unit of length in the Yee lattice.  Most MEEP users interpret this unit of length to be 1 $\mu$m because the applications are for photonics.  For example, a \textit{relative} frequency (unit-less) of 0.5 corresponds to a \textit{relative} wavelength of 2.  When interpreted as 2 $\mu$m, the frequency is 150 THz in real units that correspond to optical bandwidth.  If we choose to interpret the \textit{relative} wavelength as 2 cm, the real frequency is 15 GHz.  A \textit{relative} frequency of 0.05 corresponds to the RF frequency 1.5 GHz.  Thus, we have re-purposed MEEP from photonics simulator to RF simulator.  \\ \vspace{2.5mm}

\begin{figure}
\centering
%\includegraphics[width=0.35\textwidth]{figures/Oct30_plot2.png}
\includegraphics[width=0.33\textwidth]{figures/Oct30_plot1.png}
\includegraphics[width=0.33\textwidth]{figures/Aug11_plot2.png}
\includegraphics[width=0.33\textwidth]{figures/Aug11_plot1.png}
\caption{\label{fig:pa_1} (Left) The beam angle $\Delta \phi$ divided by the beam width $BW$ for the $N = 16$ one-dimensional Yagi array versus $\Delta \Phi$, the phase shift per element. The gray line represents theoretical expectation, and the black line is a linear fit to the data.  (Middle) $\Delta \phi$ versus $\Delta \Phi$ for the $N=16$ version of the one-dimensional horn array, for several frequencies.  (Right) The dependence of the beam width on frequency for the one-dimensional $N=16$ horn array.  The black line is a functional fit to the data $f(x) = a/x + b$ with $a=12.0\pm 0.1$ degree GHz, and $b=1.1\pm 0.2$ degrees.}
\end{figure}

By Fall 2020, we were producing CEM models using MEEP that matched expected phased array properties.  For a one-dimensional array with $N$ RF elements, there is a linear relationship between the radiated plane-wave direction $\Delta \phi$, and the phase shift per RF element $\Delta \Phi$.  The coefficient of the relationship is determined by the ratio of real wavelength to RF element spacing.  Figure \ref{fig:pa_1} contains results for our first phased array models in which the single RF elements were Yagi-Uda style antennas and horn antennas.  The linear relationship is evident in the data.  The radiated signal direction $\Delta \phi$ is divided by the beam width (BW) in Fig. \ref{fig:pa_1} (left), and left in degrees in Fig. \ref{fig:pa_1} (middle).  A beam width of a radiation pattern is the angular width of the main lobe of radiation, outside of which the radiated power has decreased by 3 dB.  In Fig. \ref{fig:pa_1} (left), the $N=16$ Yagi array can steer a 5 GHz plane wave up to four beam widths to the right or left of the forward direction.  Yagi-Uda style antennas are designed for a single frequency.  In Fig. \ref{fig:pa_1} (middle), results are shown for an $N=16$ array of horn antennas.  Since horn antennas are broadband radiators, the linear relationship is shown for 0.3, 1.5, and 3.0 GHz.  The beam width is inversely related to frequency, so $\Delta \phi$ was left in degrees.  In Fig. \ref{fig:pa_1} (right), inverse relationship between beam width and real frequency is shown. \\ \vspace{2.5mm}

We can also produce phased array radiation patterns with MEEP that match theoretical expectations.  The radiation pattern of a one-dimensional array of $N$ radiating point sources can be derived using first principles \cite{electronics10040415}.  The \textit{pattern multiplication theorem} states that a one-dimensional phased array radiation pattern of $N$ identical RF elements will be that of a row of $N$ point sources, multiplied by the radiation pattern of the individual RF element.  In Fig. \ref{fig:1dhornresults2} (left and middle), the radiated field of a $N=16$ horn array is shown in the E-plane (x-y plane).  The radiation pattern is shown in \ref{fig:1dhornresults2} (right).  The main lobe is steered 9 degrees above the x-axis, matching the theoretical expectation.  The blue curve in the polar plot represents the CEM radiation pattern from MEEP, while the red curve is the theoretical expectation from a row of $N$ point sources.  The row of point sources is symmetric, creating a back lobe at $\Delta \phi = 171$ degrees.  The horn array has no back lobe because the individual horns suppress backward radiation, as expected from the pattern multiplication theorem.  We also showed that two-dimensional arrays of Yagi-Uda and horn antennas matched theoretical expectations exactly.  Our revelation that the photonics code MEEP could be applied to phased array design led to the work receiving Top 10 honors for December 2020 to May 2021 from the editors of \textit{Electronics Journal}. \\ \vspace{2.5mm}

\begin{figure}
\centering
\includegraphics[width=5.625cm,angle=90]{figures/fields/colorbar.pdf}
%\includegraphics[width=3cm]{figures/fields/ey_phase_horn_t15.png}
\includegraphics[width=3cm]{figures/fields/ey_phase_horn_t30.png}
%\includegraphics[width=3cm]{figures/fields/ey_phase_horn_t45.png}
\includegraphics[width=3cm]{figures/fields/ey_phase_horn_t60.png}
\includegraphics[width=6cm]{figures/fields/rad_patt_field.png}
\caption{\label{fig:1dhornresults2} (Left) The $N = 16$ one-dimensional horn array, radiating a linearly polarized electric field $|\vec{E}(x,y,t)|$ at $t = 1$ ns into the simulation run, and (middle) at $t = 2.0$ ns into the run.  The 2D area is $80 \times 150$ cm$^2$.  The frequency is 2.5 GHz, and the beam angle is $\Delta \phi = 9$ degrees from broadside (x-direction). (Right) The normalized radiated power versus $\Delta \phi$ known as the radiation pattern, in dB.  The blue curve represents the results from MEEP, and the red curve is the theoretical expectation from $N$ point sources.}
\end{figure}

In the same work, we showed that MEEP can be used to model the behavior of phased arrays in realistic polar ice environments.  Most commercial CEM packages assume a uniform ground plane and index of refraction in the medium surrounding the array.  By contrast, MEEP gives the user fine control of the index of refraction of each voxel, $n(x,y,z)$.  The RF index of refraction in polar ice is $n = 1.78$ for solid ice, but varies with the depth near the snow surface.  The transitional region between surface snow and solid ice in ice shelves and sheets in polar regions is known as the \textit{firn}.  The $n(z)$ function is well-measured in a variety of locations in Antarctica \cite{horizPaper}, and Greenland \cite{deaconu_2018}.  The ARA (South Pole) \cite{PhysRevD.105.122006}, Radio Neutrino Observatory, Greenland (RNO-G) \cite{rno}, and the proposed IceCube Gen2 project (South Pole) \cite{Aartsen_2021} all use or plan to use RF phased arrays as the primary UHE-$\nu$ detector.  We propose to incorporate the actual index of refraction profile $n(z)$ into the phased array design process, which is difficult to accomplish with commerical tools. \\ \vspace{2.5mm}

The common simulation package used for ARA, RNO-G, and IceCube Gen2 is NuRadioMC, built from prior experience with ARA and ARIANNA \cite{10.1140/epjc/s10052-020-7612-8,10.1109/tns.2015.2468182,10.1016/j.astropartphys.2011.11.010,Barwick:2014pca,10.1103/physrevd.102.043021}.  NuRadioMC addresses analytically the ray-tracing solution for UHE-$\nu$ signals as they propagate through polar ice.  We derived the analytic ray-tracing solutions presented in \cite{horizPaper} and \cite{10.1140/epjc/s10052-020-7612-8}, which were adopted into NuRadioMC.  The ray-tracing approach is an approximation that does not capture the precise behavior of three-dimensional field propagation in ice with realistic properties.  A byproduct of our proposed research will be to incorporate realistic field propagation into NuRadioMC using FDTD computations.  This integration should increase the precision of the predictions made by NuRadioMC that will be checked against future ARA, RNO-G, and IceCube Gen2 data for UHE-$\nu$ interactions in polar ice volumes \cite{10.22323/1.395.1217}. \\ \vspace{2.5mm}

In Summer 2021, we again received a Faculty Fellowship from the ONR to continue this work.  Having solved the open-source CEM problem, we turned to creating more realistic 3D models of horn antennas that could be printed with 3D printers.  Two Whittier College undergraduates received research fellowships to aid in this phase of the work.  We determined how to extract schematic designs from MEEP code that can be translated into file formats usable by 3D printers.  We acquired NinjaTek 3D printer filament that was ...

%things
%
%1. Impetus for the project and the ONR
%2. Review article
%3. Decision about MEEP
%4. The first paper, and accolades.  ICRC proceeding, and n(z)
%5. 3D printing with PLA and Multi3D LLC Electrifi
%	-Examples already created
%	-The design loop
%	-GDSII file formatting, kLayout and interfacing to 3D printing

\section{The Connection to Ultra-high Energy Neutrino Observations}
\label{sec:askaryan}

Example

\section{The Connection to Remote Sensing of Ice Sheets}
\label{sec:cresis}

Example

\section{The Connection to Office of Naval Research Projects}
\label{sec:onr}

Example

\section{Conclusion, Intellectual Merits}
\label{sec:conc_im}

Example

\end{document}
